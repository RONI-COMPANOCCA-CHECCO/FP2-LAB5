%package list
\documentclass{article}
\usepackage[top=3cm, bottom=3cm, outer=3cm, inner=3cm]{geometry}
\usepackage{multicol}
\usepackage{graphicx}
\usepackage{url}
%\usepackage{cite}
\usepackage{hyperref}
\usepackage{array}
%\usepackage{multicol}
\newcolumntype{x}[1]{>{\centering\arraybackslash\hspace{0pt}}p{#1}}
\usepackage{natbib}
\usepackage{pdfpages}
\usepackage{multirow}
\usepackage[normalem]{ulem}
\useunder{\uline}{\ul}{}
\usepackage{svg}
\usepackage{xcolor}
\usepackage{listings}
\lstdefinestyle{ascii-tree}{
    literate={├}{|}1 {─}{--}1 {└}{+}1 
  }
\lstset{basicstyle=\ttfamily,
  showstringspaces=false,
  commentstyle=\color{red},
  keywordstyle=\color{blue}
}
%\usepackage{booktabs}
\usepackage{caption}
\usepackage{subcaption}
\usepackage{float}
\usepackage{array}

\newcolumntype{M}[1]{>{\centering\arraybackslash}m{#1}}
\newcolumntype{N}{@{}m{0pt}@{}}


%%%%%%%%%%%%%%%%%%%%%%%%%%%%%%%%%%%%%%%%%%%%%%%%%%%%%%%%%%%%%%%%%%%%%%%%%%%%
%%%%%%%%%%%%%%%%%%%%%%%%%%%%%%%%%%%%%%%%%%%%%%%%%%%%%%%%%%%%%%%%%%%%%%%%%%%%
\newcommand{\itemEmail}{rcompanocca@unsa.edu.pe}
\newcommand{\itemStudent}{Roni Companocca Checco}
\newcommand{\itemCourse}{Programación}
\newcommand{\itemCourseCode}{20210558}
\newcommand{\itemSemester}{II}
\newcommand{\itemUniversity}{Universidad Nacional de San Agustín de Arequipa}
\newcommand{\itemFaculty}{Facultad de Ingeniería de Producción y Servicios}
\newcommand{\itemDepartment}{Departamento Académico de Ingeniería de Sistemas e Informática}
\newcommand{\itemSchool}{Escuela Profesional de Ingeniería de Sistemas}
\newcommand{\itemAcademic}{2023 - B}
\newcommand{\itemInput}{Del 29 Septiembre 2023}
\newcommand{\itemOutput}{Al 29 Septiembre 2023}
\newcommand{\itemPracticeNumber}{05}
\newcommand{\itemTheme}{ArrayList}
%%%%%%%%%%%%%%%%%%%%%%%%%%%%%%%%%%%%%%%%%%%%%%%%%%%%%%%%%%%%%%%%%%%%%%%%%%%%
%%%%%%%%%%%%%%%%%%%%%%%%%%%%%%%%%%%%%%%%%%%%%%%%%%%%%%%%%%%%%%%%%%%%%%%%%%%%

\usepackage[english,spanish]{babel}
\usepackage[utf8]{inputenc}
\AtBeginDocument{\selectlanguage{spanish}}
\renewcommand{\figurename}{Figura}
\renewcommand{\refname}{Referencias}
\renewcommand{\tablename}{Tabla} %esto no funciona cuando se usa babel
\AtBeginDocument{%
	\renewcommand\tablename{Tabla}
}

\usepackage{fancyhdr}
\pagestyle{fancy}
\fancyhf{}
\setlength{\headheight}{30pt}
\renewcommand{\headrulewidth}{1pt}
\renewcommand{\footrulewidth}{1pt}
\fancyhead[L]{\raisebox{-0.2\height}{\includegraphics[width=3cm]{logo_episunsa.png}}}
\fancyhead[C]{\fontsize{7}{7}\selectfont	\itemUniversity \\ \itemFaculty \\ \itemDepartment \\ \itemSchool \\ \textbf{\itemCourse}}
\fancyhead[R]{\raisebox{-0.2\height}{\includegraphics[width=1.2cm]{abet.png}}}
\fancyfoot[L]{Estudiante Roni Companocca Checco}
\fancyfoot[C]{\itemCourse}
\fancyfoot[R]{Página \thepage}

% para el codigo fuente
\usepackage{listings}
\usepackage{color, colortbl}
\definecolor{dkgreen}{rgb}{0,0.6,0}
\definecolor{gray}{rgb}{0.5,0.5,0.5}
\definecolor{mauve}{rgb}{0.58,0,0.82}
\definecolor{codebackground}{rgb}{0.95, 0.95, 0.92}
\definecolor{tablebackground}{rgb}{0.8, 0, 0}

\lstset{frame=tb,
	language=bash,
	aboveskip=3mm,
	belowskip=3mm,
	showstringspaces=false,
	columns=flexible,
	basicstyle={\small\ttfamily},
	numbers=none,
	numberstyle=\tiny\color{gray},
	keywordstyle=\color{blue},
	commentstyle=\color{dkgreen},
	stringstyle=\color{mauve},
	breaklines=true,
	breakatwhitespace=true,
	tabsize=3,
	backgroundcolor= \color{codebackground},
}

\begin{document}
	
	\vspace*{10px}
	
	\begin{center}	
		\fontsize{17}{17} \textbf{ Informe de Laboratorio \itemPracticeNumber}
	\end{center}
	\centerline{\textbf{\Large Tema: \itemTheme}}
	%\vspace*{0.5cm}	

	\begin{flushright}
		\begin{tabular}{|M{2.5cm}|N|}
			\hline 
			\rowcolor{tablebackground}
			\color{white} \textbf{Nota}  \\
			\hline 
			     \\[30pt]
			\hline 			
		\end{tabular}
	\end{flushright}	

	\begin{table}[H]
		\begin{tabular}{|x{4.7cm}|x{4.8cm}|x{4.8cm}|}
			\hline 
			\rowcolor{tablebackground}
			\color{white} \textbf{Estudiante} & \color{white}\textbf{Escuela}  & \color{white}\textbf{Asignatura}   \\
			\hline 
			{\itemStudent \par \itemEmail} & \itemSchool & {\itemCourse \par Semestre: \itemSemester \par Código: \itemCourseCode}     \\
			\hline 			
		\end{tabular}
	\end{table}		
	
	\begin{table}[H]
		\begin{tabular}{|x{4.7cm}|x{4.8cm}|x{4.8cm}|}
			\hline 
			\rowcolor{tablebackground}
			\color{white}\textbf{Laboratorio} & \color{white}\textbf{Tema}  & \color{white}\textbf{Duración}   \\
			\hline 
			\itemPracticeNumber & \itemTheme & 02 horas   \\
			\hline 
		\end{tabular}
	\end{table}
	
	\begin{table}[H]
		\begin{tabular}{|x{4.7cm}|x{4.8cm}|x{4.8cm}|}
			\hline 
			\rowcolor{tablebackground}
			\color{white}\textbf{Semestre académico} & \color{white}\textbf{Fecha de inicio}  & \color{white}\textbf{Fecha de entrega}   \\
			\hline 
			\itemAcademic & \itemInput &  \itemOutput  \\
			\hline 
		\end{tabular}
	\end{table}

    \section{TAREA}
	\begin{itemize}	
    \subsection{Objetivos:}
		\item Crear e inicializar ArrayList
		\item Realizar búsquedas secuencial y binaria en un ArrayList
        \item Implementar métodos de ordenamiento en ArrayList
        \item Solucionar problemas
    \subsection{Competencias a alcanzar:}
		\item Diseña, responsablemente, sistemas, componentes o procesos para satisfacer necesidades dentro de restricciones realistas: económicas, medio ambientales, sociales, políticas, éticas, de salud, de seguridad, manufacturación y sostenibilidad.
        \item Aplica de forma flexible, técnicas, métodos, principios, normas, estándares y herramientas de ingeniería necesarias para la construcción de software e implementación de sistemas de información.

    \section{EQUIPOS, MATERIALES Y TEMAS UTILIZADOS}
	\begin{itemize}
		\item Sistema Operativo Windows
		\item OpenJDK 64-Bits 17.0.7.
		\item Git 2.39.2.	
  	\item Cuenta en GitHub con el correo institucional.
	\end{itemize}

    \section{URL DE REPOSITORIO GITHUB}
	\begin{itemize}
		\item URL para el Repositorio GitHub.
		\item \url{https://github.com/RONI-COMPANOCCA-CHECCO}
		\item URL para el laboratorio 05 en el Repositorio GitHub.	
        \item \url{https://github.com/RONI-COMPANOCCA-CHECCO/FP2-LAB5}
	\end{itemize}
    
    \section{ACTIVIDADES}
	\begin{itemize}
        \subsection{EJERCICIO RESUELTO}
		\item - Creamos la Clase ColeccionesList con los siguientes métodos que realizan las operaciones que se pueden realizar con el ArrayList:

        \begin{lstlisting}[language=java]
// RONI COMPANOCCA CHECCO
// CUI: 20210558
// LABORATORIO 05
// FUNDAMENTOS DE PROGRAMACION 
import java.util.List;
import java.util.ArrayList;
import java.util.Iterator;
import java.util.Scanner;

public class ColeccionesList{
    static Scanner consola = new Scanner(System.in);
    // implementacion de la interfaz
    List<String> listaCadenas = new ArrayList<String>();
    
    public void llenarArrayList(){
        // agregamos elementos
        listaCadenas.add("Arequipa");
        listaCadenas.add("Tacna");
        listaCadenas.add("Moquegua");
        listaCadenas.add("Trujillo");
        listaCadenas.add("Lima");
        listaCadenas.add("Chiclayo");
        listaCadenas.add("Cajamarca");
        // permite duplicados
        listaCadenas.add("Cajamarca");
        listaCadenas.add("Cuzco");
        listaCadenas.add("Arequipa");
        Iterator iterador = listaCadenas.iterator();
        // recorremos y mostramos contenido de lista
        System.out.println(("\nARRAYLIST\n"));
        while(iterador.hasNext()){
            String elemento = (String) iterador.next();
            System.out.println(elemento);
        }
        // muestra el tamaño de la lista
        System.out.println("Tamaño: "+listaCadenas.size());
        // retorna un objeto segun su posicion
        int indice = listaCadenas.indexOf("Arequipa");
        System.out.println("Elemento: "+listaCadenas.get(indice)+"Posicion: "+indice);
        indice = listaCadenas.indexOf("Lima");
        System.out.println("Elemento: "+listaCadenas.get(indice)+"Posicion: "+indice);
    }
        
    public void buscarEnArrayList(){
        //busqueda de un objeto dentro de la coleccion
        System.out.println("Ingrese un valor a Buscar: ");
        String objetoBuscar = consola.next();

        if(listaCadenas.contains(objetoBuscar)){
            System.out.println(objetoBuscar+" Se encuentra en la Lista");
        }else {
            System.out.println(objetoBuscar+" No se encuentra!!");
        }
        
    }

    public void eliminarEnArrayList(){
        //eliminar un objeto de la coleccion
        System.out.println("Ingrese un valor a Borrar: ");
        String objetoBorrar1 = consola.next();
        int posicion = listaCadenas.indexOf(objetoBorrar1);
        listaCadenas.remove(posicion);

        //eliminar todos los objeto de la coleccion
        listaCadenas.clear();
        if (listaCadenas.isEmpty()){
            System.out.println("La lista se encuentra vacia");
        }
    }
}
        \end{lstlisting}
		\item Segundo: Creamos la Clase Main, donde llamamos a nuestra clase y las operaciones implementadas:
        \begin{lstlisting}[language=java]
public class ArrayList01 {
    public static void main(String[] args){
        ColeccionesList coleccion01 = new ColeccionesList();
        System.out.println("--------------------");
        System.out.println("COLECCION ARRAYLIST");
        System.out.println("--------------------");
        coleccion01.llenarArrayList();
        coleccion01.buscarEnArrayList();
        coleccion01.eliminarEnArrayList();
    }
    
}
        \end{lstlisting}
        \item Ejecucion
        \begin{lstlisting}[language=java]
--------------------
COLECCION ARRAYLIST
--------------------

ARRAYLIST

Arequipa
Tacna
Moquegua
Trujillo
Lima
Chiclayo
Cajamarca
Cajamarca
Cuzco
Arequipa
Tamaño: 10
Elemento: ArequipaPosicion: 0
Elemento: LimaPosicion: 4
Ingrese un valor a Buscar:
4
4 No se encuentra!!
Ingrese un valor a Borrar:
4
        \end{lstlisting}

        \subsection{EJERCICIO PROPUESTO}
        \item Cree un Proyecto llamado Laboratorio5
        \item Usted deberá crear las dos clases Soldado.java y VideoJuego3.java. Puede reutilizar lo desarrollado en Laboratorios anteriores.
        \item Del Soldado nos importa el nombre, puntos de vida, fila y columna (posición en el tablero).
        \item El juego se desarrollará en el mismo tablero de los laboratorios anteriores. Pero ahora el tablero debe ser un ArrayList bidimensional.
        \item Tendrá 2 Ejércitos. Inicializar el tablero con n soldados aleatorios entre 1 y 10 para cada Ejército. Cada soldado tendrá un nombre autogenerado: Soldado0X1, Soldado1X1, etc., un valor de puntos de vida autogenerado aleatoriamente [1..5], la fila y columna también autogenerados aleatoriamente (no puede haber 2 soldados en el mismo cuadrado). Se debe mostrar el tablero con todos los soldados creados (distinguir los de un ejército de los del otro ejército). Además de los datos del Soldado con mayor vida de cada ejército, el promedio de puntos de vida de todos los soldados creados por ejército, los datos de todos los soldados por ejército en el orden que fueron creados y un ranking de poder de todos los soldados creados por ejército (del que tiene más nivel de vida al que tiene menos) usando 2 diferentes algoritmos de ordenamiento. Finalmente, que muestre qué ejército ganará la batalla (indicar la métrica usada para decidir al ganador de la batalla).
        
        \item la clase VideoJuego3.java        
        \begin{lstlisting}[language=java]
// RONI COMPANOCCA CHECCO
// CUI: 20210558
// LABORATORIO 05
// FUNDAMENTOS DE PROGRAMACION 
import java.util.*;

public class VideoJuego3 {

	public static void main(String[] args) {

		//DECLARACION DE VARIABLES Y ARREGLOS NECESARIOS
		ArrayList<Soldado> ejercito1 = new ArrayList();
		ArrayList<Soldado> ejercito2 = new ArrayList();
		ArrayList<ArrayList<Soldado>> tablero = new ArrayList();
		int batallon1, batallon2;
		int vidatotal1=0, vidatotal2=0;
		double promedioVida1=0, promedioVida2=0;

		// BUCLE PARA DESIGNAR LA CANTIDAD DE FILAS Y COLUMNAS DEL TABLERO
		for(int i=0; i<10; i++) {
			tablero.add(new ArrayList<Soldado>());
			for(int j=0; j<10; j++) {
				tablero.get(i).add(new Soldado());
			}
		}

		// CREACION DEL NUMERO DE POSICIONES DE CADA EJERCITO
		batallon1 = aleatorio(1,10);
		batallon2 = aleatorio(1,10);

		// INICIALIZAR ARREGLOS
		inicializarArreglo(ejercito1, batallon1);
		inicializarArreglo(ejercito2, batallon2);

		// GENERAR EJERCITOS VALIDOS
		generarEjercitos(ejercito1, ejercito2);

		// AÑADIR LOS EJERCITOS AL TABLERO
		añadirTablero(ejercito1, tablero);
		añadirTablero(ejercito2, tablero);

		//IMPRIMIR EL TABLERO
		imprimirTablero(tablero);

		//IMPRIMIR LOS SOLDADOS DE MAYOR VIDA DE CADA EJERCITO
		System.out.println("Soldado de mayor vida del ejercito 1");
		SoldadoConMayorVida(ejercito1);
		System.out.println("soldado de mayor vida del ejercito 2");
		SoldadoConMayorVida(ejercito2);

		//IMPRIMIR LA VIDA TOTAL Y EL PROMEDIO DEL EJERCITO 1
		System.out.println("\nEJERCITO 1: ");
		for (int i=0; i<ejercito1.size(); i++) {
			vidatotal1+=ejercito1.get(i).getPuntos();
			promedioVida1 = vidatotal1/(ejercito1.size()*1.0);
		}
		System.out.println("Vida total: "+vidatotal1);
		System.out.println("Promedio de vida: "+promedioVida1);
		
		//IMPRIMIR LA VIDA TOTAL Y EL PROMEDIO DEL EJERCITO 2
		System.out.println("\nEJERCITO 2: ");
		for (int i=0; i<ejercito2.size(); i++) {
			vidatotal2+=ejercito2.get(i).getPuntos();
			promedioVida2 = vidatotal2/(ejercito2.size()*1.0);
		}
		System.out.println("Vida total: "+vidatotal2);
		System.out.println("Promedio de vida: "+promedioVida2);

		//IMPRIMIR LOS SOLDADOS CREADOS EN EL ORDEN POR DEFECTO
		System.out.println("\nLista ejercito 1:");
		for(int i=0; i<ejercito1.size(); i++) {
			imprimir(ejercito1.get(i));
		}
		System.out.println("\nLista ejercito 2:");
		for(int i=0; i<ejercito2.size(); i++) {
			imprimir(ejercito2.get(i));
		}

		// IMPRIMIR LOS DATOS DE LOS SOLDADOS ORDENADOS DE MAYOR A MENOR DEPENDIENDO DE SU NIVEL DE VIDA USANDO DOS TIPOS DE ALGORITMO
		ordenarPorVidaMetodoA(ejercito1);
		ordenarPorVidaMetodoB(ejercito2);
		System.out.println("\nEjercito 1 Ordenados por nivel de vida");
		for(int i=0; i<ejercito1.size(); i++) {
			imprimir(ejercito1.get(i));
		}
		System.out.println("\nEjercito 2 Ordenados por nivel de vida");
		for(int i=0; i<ejercito2.size(); i++) {
			imprimir(ejercito2.get(i));
		}
		
		// MOSTRAR EJERCITO GANADOR LA METRICA USADA ARA DESIGNAR AL GANADOR ES POR EL NIVEL DEL PROMEDIO DE VIDA DE CADA EJERCITO
		if(promedioVida1>promedioVida2) {
			System.out.println("\nGANADOR ***EJERCITO 1***");
		}else if (promedioVida1<promedioVida2) {
			System.out.println("\nGANADOR ***EJERCITO 2***");
			}
		else {
			System.out.print("\n***ES UN EMPATE***");
		}
	}

	// METODO PARA CREAR NUMEROS ALEATORIOS EN UN RANGO
	public static int aleatorio(int min, int max) {
		return(int)(Math.random()*(max-min+1)+min);
	}

	// METODO PARA INICIAR UN ARRAYLIST
	public static void inicializarArreglo (ArrayList<Soldado> soldadito, int num) {
		for (int i=0; i<num; i++) {
			soldadito.add(new Soldado());
		}
	}

	// METODO PARA GENERAR DATOS DEL OBJETO SOLDADO
	public static Soldado generarDatos() {
		Soldado soldadito = new Soldado();
		soldadito.setPuntos(aleatorio(1,5));
		soldadito.setColumna(aleatorio(1,10));
		soldadito.setFila(aleatorio(1,10));
		return soldadito;
	}

	// METODOS PARA GENERAR LOS EJERCITOS DE MANERA ALEATORIA
	public static void generarEjercitos(ArrayList<Soldado>B1, ArrayList<Soldado>B2) {
		ArrayList<Soldado>Soldados = new ArrayList();
		Soldados.add(generarDatos());
		for (int i=1; i<(B1.size()+B2.size()); i++) {
			Soldados.add(generarDatos());
			for (int j=0; j<i; j++) {
				if(Soldados.get(i).getFila()==Soldados.get(j).getFila()) {
					if(Soldados.get(i).getColumna()==Soldados.get(j).getColumna()){
						Soldados.remove(i);
						i--;
					}
				}
			}
		}
		for (int i=0; i<B1.size(); i++) {
			B1.add(i, Soldados.get(i));
			B1.get(i).setNombre("Soldado"+i+"x1");
			B1.get(i).setColumn(B1.get(i).getPuntos()+"[E1]");
			B1.remove(i+1);
		}
		for (int i=0; i<B2.size(); i++) {
			B2.add(i, Soldados.get(i+B1.size()));
			B2.get(i).setNombre("Soldado"+i+"x2");
			B2.remove(i+1);
			B2.get(i).setColumn(B2.get(i).getPuntos()+"[E2]");
		}
	}

	// METODO PARA AÑADIR LOS EJERCITOS AL TABLERO
	public static void añadirTablero(ArrayList<Soldado>soldadito, ArrayList<ArrayList<Soldado>>table) {
		for (int i=0; i<soldadito.size(); i++) {
			table.get(soldadito.get(i).getColumna()-1).add(soldadito.get(i).getFila()-1,soldadito.get(i));
			table.get(soldadito.get(i).getColumna()-1).remove(soldadito.get(i).getFila());
		}
	}

	// METODO PARA IMPRIMIR EL TABLERO EN LA CUAL SE DESARROLLA EL JUEGO
	public static void imprimirTablero(ArrayList<ArrayList<Soldado>> table) {
		System.out.println("\tA\tB\tC\tD\tF\tG\tH\tI\tJ");
		for(int i=0; i<table.size(); i++) {
			System.out.print(i+1);
			for(int j=0; j<table.get(i).size();j++) {
				System.out.print("\t"+table.get(i).get(j).getColumn());
			}
			System.out.println("\n");
		}
	}
	
	//METODO PARA IMPRIMIR LOS SOLDADOS DE MAYOR VIDA
	public static void SoldadoConMayorVida (ArrayList<Soldado>soldadito) {
		Soldado mayor = new Soldado();
		mayor.setPuntos(0);
		for(int i=0; i<soldadito.size();i++) {
			if (mayor.getPuntos()<soldadito.get(i).getPuntos()) {
				mayor = soldadito.get(i);
			}
		}
		imprimir(mayor);
	}
	
	// METODO PARA IMPRIMIR EL NOMBRE, LA POSICION Y NIVEL DE VIDA DEL SOLDADO
	public static void imprimir(Soldado soldadito) {
		System.out.println("Nombre: "+soldadito.getNombre()+"\nPosicion: "+soldadito.getColumna()+"X"+soldadito.getFila()+"\tVida: "+soldadito.getPuntos());
	}
	
	// METODO QUE NOS AYUDA A ORDENAR LOS SOLDADOS DE ACUERDO A SU NIVEL DE VIDA, USUANDO UN ALGORITMO DE ORDENAMIENTO DE BURBUJA
	public static void ordenarPorVidaMetodoA(ArrayList<Soldado>soldadito) {
		Soldado aux = new Soldado();
		for(int i=0; i<soldadito.size()-1; i++) {
			for(int j=0; j<soldadito.size()-i-1; j++) {
				if(soldadito.get(j).getPuntos()<soldadito.get(j+1).getPuntos()) {
					aux = soldadito.get(j);
					soldadito.set(j,soldadito.get(j+1));
					soldadito.set(j+1,aux);
				}
			}
		}
	}

    // METODO QUE NOS AYUDA A ORDENAR LOS SOLDADOS DE ACUERDO A SU NIVEL DE VIDA, EN ESTA OCACION DIFERENTE A LA ANTERIOR QUE ERA ALGORITMO DE BURBUJA
    public static void ordenarPorVidaMetodoB(ArrayList<Soldado> soldadito) {
        Collections.sort(soldadito, new Comparator<Soldado>() {
            public int compare(Soldado s1, Soldado s2) {
                // Orden descendente por puntos de vida
                return Integer.compare(s2.getPuntos(), s1.getPuntos());
            }
        });
    }
}
        \end{lstlisting}

        \item la clase Soldado.java
        \begin{lstlisting}[language=java]
// RONI COMPANOCCA CHECCO
// CUI: 20210558
// LABORATORIO 05
// FUNDAMENTOS DE PROGRAMACION - laboratorio
// clase Soldado.java
public class Soldado {
	private String nombre;
	private int fila;
	private int columna;
	private int puntos;
	private String column;
	
	public Soldado() {
		nombre = "";
		fila = 0;
		columna = 0;
		puntos = 0;
		column = "";
	}

	// METODOS MUTADORES
	public void setNombre(String nombre) {
		this.nombre = nombre;
	}

	public void setFila(int fila) {
		this.fila = fila;
	}

	public void setColumna(int columna) {
		this.columna = columna;
	}

	public void setPuntos(int puntos) {
		this.puntos = puntos;
	}
	
	public void setColumn(String column) {
		this.column = column;
	}

	// METODOS ACCESORES
	public String getNombre() {
		return nombre;
	}

	public int getFila() {
		return fila;
	}

	public int getColumna() {
		return columna;
	}

	public int getPuntos() {
		return puntos;
	}
	
	public String getColumn() {
		return column;
	}
}
         \end{lstlisting}

         \item Ejecucion
         \begin{lstlisting}[language=java]
        A       B       C       D       F       G       H       I       J
1

2                       1[E2]

3

4

5                       3[E2]                   1[E1]

6               2[E2]                   2[E2]   2[E2]

7                       5[E2]                   4[E2]

8

9

10

Soldado de mayor vida del ejercito 1
Nombre: Soldado0x1
Posicion: 5X6   Vida: 1
soldado de mayor vida del ejercito 2
Nombre: Soldado4x2
Posicion: 7X3   Vida: 5

EJERCITO 1:
Vida total: 1
Promedio de vida: 1.0

EJERCITO 2:
Vida total: 19
Promedio de vida: 2.7142857142857144

Lista ejercito 1:
Nombre: Soldado0x1
Posicion: 5X6   Vida: 1

Lista ejercito 2:
Nombre: Soldado0x2
Posicion: 6X5   Vida: 2
Nombre: Soldado1x2
Posicion: 2X3   Vida: 1
Nombre: Soldado2x2
Posicion: 6X2   Vida: 2
Nombre: Soldado3x2
Posicion: 6X6   Vida: 2
Nombre: Soldado4x2
Posicion: 7X3   Vida: 5
Nombre: Soldado5x2
Posicion: 5X3   Vida: 3
Nombre: Soldado6x2
Posicion: 7X6   Vida: 4

Ejercito 1 Ordenados por nivel de vida
Nombre: Soldado0x1
Posicion: 5X6   Vida: 1

Ejercito 2 Ordenados por nivel de vida
Nombre: Soldado4x2
Posicion: 7X3   Vida: 5
Nombre: Soldado6x2
Posicion: 7X6   Vida: 4
Nombre: Soldado5x2
Posicion: 5X3   Vida: 3
Nombre: Soldado0x2
Posicion: 6X5   Vida: 2
Nombre: Soldado2x2
Posicion: 6X2   Vida: 2
Nombre: Soldado3x2
Posicion: 6X6   Vida: 2
Nombre: Soldado1x2
Posicion: 2X3   Vida: 1

GANADOR ***EJERCITO 2***
         \end{lstlisting}
 
    \section{CUESTIONARIO}

	\begin{itemize}
	\subsection {¿Cómo se declara e inicializa un ArrayList bidimensional?}
        \item En Java, no existe un tipo de dato ArrayList bidimensional como tal. Sin embargo, puedes lograr una funcionalidad similar utilizando un ArrayList de ArrayLists, lo que te permitirá crear una estructura de datos bidimensional.
        \begin{lstlisting}[language=java]
        ArrayList<ArrayList<Integer>> matriz = new ArrayList<>();
        \end{lstlisting}
		
    \subsection {¿Qué ventajas tienen los ArrayList con respecto a los arreglos?}
        \item Los ArrayList en Java tienen varias ventajas con respecto a los arreglos (arrays) que los hacen una elección más flexible y poderosa en muchas situaciones: Tamaño dinámico, Inserción y eliminación eficiente, Flexibilidad, Métodos útiles, Uso de autoboxing, etc.

	\subsection {¿Mencione 4 métodos importantes del ArrayList? ¿Qué finalidad tienen?}
        \item add(E elemento): Este método se utiliza para agregar un elemento al final del ArrayList. La finalidad principal de este método es aumentar la capacidad del ArrayList si es necesario y agregar el elemento especificado al final de la lista.
        \item remove(int índice): Este método se usa para eliminar el elemento en el índice especificado del ArrayList. La finalidad principal es eliminar un elemento en una posición particular de la lista.
        \item get(int índice): Este método se emplea para obtener el elemento en el índice especificado del ArrayList. La finalidad es recuperar un elemento en una posición particular de la lista.
        \item size(): Este método devuelve el número de elementos en el ArrayList. La finalidad es conocer la cantidad de elementos presentes en la lista.
	\end{itemize}
	
	\section{REFERENCIAS}
	\begin{itemize}
		\item M. Aedo, “Fundamentos de Programación 2 - Tópicos de Programación Orientada a Objetos”, Primera Edición, 2021, Editorial UNSA.
		\item \url{https://github.com/rescobedoq/programacion.git}
		\item J. Dean, "Introduction to programming with Java: A Problem Solving Approach”, Third Edition, 2021, McGraw-Hill.
        \item C. T. Wu, "An Introduction to Object-Oriented Programming with Java", Fifth Edition, 2010, McGraw-Hill.
        \item P. Deitel, "Java How to Program", Eleventh Edition, 2017, Prentice Hall.
	\end{itemize}
	
%\clearpage
%\bibliographystyle{apalike}
%\bibliographystyle{IEEEtranN}
%\bibliography{bibliography}
			
\end{document}